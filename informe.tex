\documentclass[12pt,a4paper]{article}

% ======== PAQUETES ========
\usepackage[spanish]{babel}     % Idioma español
\usepackage[utf8]{inputenc}     % Acentos
\usepackage[T1]{fontenc}        % Caracteres correctos
\usepackage{amsmath, amssymb}   % Matemática avanzada
\usepackage{graphicx}           % Insertar imágenes
\usepackage{float}              % Control de posición de figuras
\usepackage{hyperref}           % Hipervínculos
\usepackage{geometry}           % Márgenes
\usepackage{listings}           % Código fuente
\usepackage{xcolor}             % Colores para código
\usepackage{caption}            % Mejoras en captions
\usepackage{subcaption}         % Subfiguras
\usepackage{booktabs}           % Tablas bonitas

% Configuración de márgenes
\geometry{top=2.5cm, bottom=2.5cm, left=2.0cm, right=2.0cm}

% Eliminar sangría en todos los párrafos
\setlength{\parindent}{0pt}

% Configuración del código fuente
\lstset{
    language=Python,
    backgroundcolor=\color{black!5},
    basicstyle=\ttfamily\footnotesize,
    keywordstyle=\color{blue},
    commentstyle=\color{green!50!black},
    stringstyle=\color{orange},
    numbers=left,
    numberstyle=\tiny,
    stepnumber=1,
    frame=single,
    breaklines=true,
    tabsize=4
}

% Datos de la portada
\title{\textbf{Métodos Computacionales - Trabajo Práctico 1}\\
Resoluciones Numéricas de Ecuaciones Diferenciales}
\author{
    Mariño Martina, Martinez Kiara\\ 
    Universidad Torcuato Di Tella
}
\date{\today}

% ======== DOCUMENTO ========
\begin{document}

% ======== PORTADA ========
\maketitle
\thispagestyle{empty}
\newpage

% ======== ÍNDICE ========
\tableofcontents
\newpage

% ======== SECCIONES ========

\part{Ecuación del Calor}

\section{Formulación matemática - Derivación de los métodos explícito e implícito}

En esta sección vamos a ver cómo se llega a las fórmulas de los métodos explícito e implícito para resolver la ecuación del calor usando diferencias finitas.  
La idea es empezar de la ecuación original, discretizarla, y después aproximar las derivadas.

\subsection{La ecuación que queremos resolver}

La ecuación del calor en una dimensión es:
\[
\frac{\partial u}{\partial t} = \alpha \frac{\partial^2 u}{\partial x^2},
\quad x \in (0,1), \quad t > 0
\]

donde:
\begin{itemize}
    \item $u(x,t)$ representa la temperatura en el punto $x$ y tiempo $t$.
    \item $\alpha > 0$ es la constante de difusión térmica.
\end{itemize}

Además, tenemos condiciones de frontera de Dirichlet:
\[
u(0,t) = 0, \quad u(1,t) = 0,
\]
y una condición inicial que nos da la distribución de temperatura al inicio:
\[
u(x,0) = f(x).
\]

En palabras, se trata de una barra de longitud 1 donde los extremos se mantienen a temperatura cero.  
Sabemos cómo estaba la temperatura al comienzo y queremos ver cómo cambia con el tiempo.

\subsection{Discretización: malla de espacio y tiempo}

Para resolver el problema de forma numérica dividimos el espacio y el tiempo en puntos separados por pasos fijos:

\begin{itemize}
    \item Dividimos el intervalo espacial $[0,1]$ en $N$ puntos con separación $\Delta x$.  
    Así, las posiciones son:
    \[
    x_j = j\Delta x, \quad j = 0,1,\dots,N
    \]
    donde $\Delta x = \tfrac{1}{N}$.
    
    \item El tiempo se divide en pasos de tamaño $\Delta t$.  
    Los instantes de tiempo quedan como:
    \[
    t_n = n\Delta t, \quad n = 0,1,2,\dots,M
    \]
    con $M\Delta t = T$ siendo el tiempo final de simulación.
\end{itemize}

Notación: llamamos $u_j^n$ a la aproximación numérica de $u(x_j,t_n)$.  
Los nodos de frontera son $j=0$ y $j=N$ (donde ya conocemos $u$ gracias a las condiciones de borde). Los nodos internos son $j=1,\dots,N-1$, que son los que vamos a actualizar.

\subsection{Aproximación de las derivadas con diferencias finitas}

Reemplazamos las derivadas por aproximaciones usando diferencias finitas:

\begin{itemize}
    \item Para la derivada temporal usamos una \textbf{diferencia hacia adelante}:
    \[
    \frac{\partial u}{\partial t}(x_j,t_n) \approx \frac{u_j^{n+1}-u_j^n}{\Delta t}.
    \]

    \item Para la segunda derivada espacial usamos \textbf{diferencias centradas}:
    \[
    \frac{\partial^2 u}{\partial x^2}(x_j,t_n) \approx \frac{u_{j+1}^n - 2u_j^n + u_{j-1}^n}{(\Delta x)^2}.
    \]
\end{itemize}

Estas aproximaciones introducen un error, pero mientras $\Delta x$ y $\Delta t$ sean pequeños, la aproximación es bastante buena.

\subsection{Método explícito}

Partimos de la ecuación del calor discretizada:
\[
\frac{u_j^{n+1}-u_j^n}{\Delta t} = \alpha \frac{u_{j+1}^n - 2u_j^n + u_{j-1}^n}{(\Delta x)^2}.
\]

\subsection*{Definición de $r$:}
Definimos un parámetro muy útil:
\[
r = \frac{\alpha \Delta t}{(\Delta x)^2}.
\]
Este parámetro combina la constante física $\alpha$ con el paso de tiempo $\Delta t$ y el paso espacial $\Delta x$.

\paragraph{Paso 1. Sustituir $r$}  
\[
u_j^{\,n+1}-u_j^{\,n} = r\big(u_{j+1}^{\,n}-2u_j^{\,n}+u_{j-1}^{\,n}\big).
\]

\paragraph{Paso 2. Aislar $u_j^{\,n+1}$}  
\[
u_j^{\,n+1} = u_j^{\,n} + r\big(u_{j+1}^{\,n}-2u_j^{\,n}+u_{j-1}^{\,n}\big).
\]

\paragraph{Paso 3. Expandir el paréntesis}  
\[
u_j^{\,n+1} = u_j^{\,n} + r\,u_{j+1}^{\,n} - 2r\,u_j^{\,n} + r\,u_{j-1}^{\,n}.
\]

\paragraph{Paso 4. Agrupar términos}  
\[
u_j^{\,n+1} = (1 - 2r)u_j^{\,n} + r\,u_{j-1}^{\,n} + r\,u_{j+1}^{\,n}.
\]

El nuevo valor de temperatura en $j$ depende del valor previo en el mismo punto y de sus dos vecinos. Es una combinación lineal:
\begin{itemize}
    \item $(1-2r)u_j^n$ (el valor previo en $j$),
    \item $r\,u_{j-1}^n$ (vecino izquierdo),
    \item $r\,u_{j+1}^n$ (vecino derecho).
\end{itemize}

\paragraph{Esquema explícito}  
\[
\boxed{u_j^{\,n+1} = (1 - 2r)u_j^{\,n} + r\,u_{j-1}^{\,n} + r\,u_{j+1}^{\,n}}
\]

\subsection*{Condición de estabilidad:}
Este método solo funciona bien si se cumple:
\[
r = \frac{\alpha \Delta t}{(\Delta x)^2} \leq \frac{1}{2}.
\]
De lo contrario, la solución numérica se vuelve inestable, con oscilaciones crecientes que no reflejan el fenómeno físico real.

\subsection*{Interpretación:}
La temperatura en $j$ tiende a ``suavizarse'' dependiendo de cómo están los vecinos $j-1$ y $j+1$. Es como un promedio ponderado que se va corrigiendo paso a paso.

\subsection{Método implícito}

Partimos de la misma aproximación temporal, pero evaluando la derivada espacial en $n+1$:
\[
\frac{u_j^{\,n+1}-u_j^{\,n}}{\Delta t}
\approx
\alpha\,\frac{u_{j+1}^{\,n+1}-2u_j^{\,n+1}+u_{j-1}^{\,n+1}}{(\Delta x)^2}.
\]

\paragraph{Paso 1. Multiplicar por $\Delta t$}  
\[
u_j^{\,n+1}-u_j^{\,n}
= \alpha\,\frac{\Delta t}{(\Delta x)^2}\big(u_{j+1}^{\,n+1}-2u_j^{\,n+1}+u_{j-1}^{\,n+1}\big).
\]

\paragraph{Paso 2. Definir $r$ y sustituir}  
\[
r := \frac{\alpha\,\Delta t}{(\Delta x)^2},
\]
\[
u_j^{\,n+1}-u_j^{\,n} = r\big(u_{j+1}^{\,n+1}-2u_j^{\,n+1}+u_{j-1}^{\,n+1}\big).
\]

\paragraph{Paso 3. Reorganizar términos}  
\[
u_j^{\,n+1} - r\big(u_{j+1}^{\,n+1}-2u_j^{\,n+1}+u_{j-1}^{\,n+1}\big) = u_j^{\,n}.
\]

\paragraph{Paso 4. Expandir el paréntesis}  
\[
u_j^{\,n+1} - r\,u_{j+1}^{\,n+1} + 2r\,u_j^{\,n+1} - r\,u_{j-1}^{\,n+1} = u_j^{\,n}.
\]

\paragraph{Paso 5. Agrupar coeficientes}  
\[
-r\,u_{j-1}^{\,n+1} + (1+2r)\,u_j^{\,n+1} - r\,u_{j+1}^{\,n+1} = u_j^{\,n}.
\]

\paragraph{Esquema implícito}  
\[
\boxed{-r\,u_{j-1}^{\,n+1} + (1+2r)\,u_j^{\,n+1} - r\,u_{j+1}^{\,n+1} = u_j^{\,n}}
\]

\subsection*{Interpretación:} 
En este caso cada valor $u_j^{\,n+1}$ está acoplado con sus vecinos en el mismo nivel temporal $n+1$.  
Por eso, no hay una fórmula directa: es necesario resolver un sistema lineal en cada paso.  
La gran ventaja es que el método es \textbf{incondicionalmente estable}, sin restricciones sobre $r$.

\subsection{Resumen comparativo}

\begin{itemize}
    \item \textbf{Explícito:} sencillo de implementar, no requiere resolver sistemas, pero exige la condición $r \leq \tfrac{1}{2}$.
    \item \textbf{Implícito:} más costoso porque requiere resolver un sistema en cada paso, pero estable para cualquier $\Delta t$ y $\Delta x$.
\end{itemize}

\section{Sistemas lineales y forma matricial}

\subsection{Recordemos el dominio}

Tenemos \(N+1\) puntos en el espacio, numerados de \(j=0\) a \(j=N\).  
Los extremos (\(j=0\) y \(j=N\)) son conocidos gracias a las condiciones de frontera \(u(0,t)=u(1,t)=0\). Los valores desconocidos corresponden a los nodos internos \(j=1,2,\dots,N-1\).  

Esto significa que en cada instante de tiempo debemos trabajar con un vector de tamaño \(N-1\), que contiene únicamente los valores de los nodos internos.

\subsection{Método explícito como sistema lineal}

El método explícito en su forma final es:
\[
u_j^{n+1} = (1-2r)u_j^n + r\,u_{j-1}^n + r\,u_{j+1}^n, 
\quad j = 1, 2, \dots, N-1.
\]

\subsection*{Forma matricial:}
Podemos escribir esta relación como:
\[
U^{n+1} = B\,U^n,
\]
donde:
\[
U^n = 
\begin{bmatrix}
u_1^n \\[4pt] u_2^n \\[4pt] \vdots \\[4pt] u_{N-2}^n \\[4pt] u_{N-1}^n
\end{bmatrix}
\]
es el vector con los valores de los nodos internos en el tiempo \(t_n\). \\

La matriz \(B\) tiene estructura tridiagonal:
\[
B = 
\begin{bmatrix}
1-2r & r     & 0     & \cdots & 0 \\[4pt]
r     & 1-2r & r     & \cdots & 0 \\[4pt]
0     & r     & 1-2r & \cdots & 0 \\[4pt]
\vdots& \vdots& \vdots& \ddots & r \\[4pt]
0     & 0     & 0     & r      & 1-2r
\end{bmatrix}.
\]

\subsection*{Interpretación:}
Para pasar del tiempo \(t_n\) al tiempo \(t_{n+1}\), basta con multiplicar el vector \(U^n\) por la matriz \(B\).  
El método explícito se reduce, por lo tanto, a una simple multiplicación matricial.

\subsection{Método implícito como sistema lineal}

En el método implícito, la ecuación para cada nodo es:
\[
-r\,u_{j-1}^{n+1} + (1+2r)\,u_j^{n+1} - r\,u_{j+1}^{n+1} = u_j^n,
\quad j = 1, 2, \dots, N-1.
\]

\subsection*{Forma matricial:} 
Esto se puede escribir como un sistema lineal:
\[
A\,U^{n+1} = U^n,
\]
donde \(U^{n+1}\) es el vector de incógnitas en el nuevo tiempo:
\[
U^{n+1} = 
\begin{bmatrix}
u_1^{n+1} \\[4pt] u_2^{n+1} \\[4pt] \vdots \\[4pt] u_{N-2}^{n+1} \\[4pt] u_{N-1}^{n+1}
\end{bmatrix}.
\] \\

La matriz \(A\) también es tridiagonal, pero con coeficientes distintos:
\[
A = 
\begin{bmatrix}
1+2r & -r    & 0     & \cdots & 0 \\[4pt]
-r    & 1+2r & -r    & \cdots & 0 \\[4pt]
0     & -r    & 1+2r & \cdots & 0 \\[4pt]
\vdots& \vdots& \vdots& \ddots & -r \\[4pt]
0     & 0     & 0     & -r     & 1+2r
\end{bmatrix}.
\]

\subsection*{Interpretación:} 
En este caso, el nuevo vector \(U^{n+1}\) no se puede calcular de forma directa, ya que está involucrado en ambos lados de la ecuación.  
Para avanzar un paso en el tiempo es necesario resolver el sistema lineal:
\[
A\,U^{n+1} = U^n.
\] \\

Esto se puede hacer en Python con:
\[
\texttt{U\_next = numpy.linalg.solve(A, U)}
\]

\subsection{Resumen comparativo}

\begin{itemize}
    \item \textbf{Explícito:}  
    \[
    U^{n+1} = B\,U^n
    \]
    Evoluciona mediante una multiplicación matricial sencilla.
    
    \item \textbf{Implícito:}  
    \[
    A\,U^{n+1} = U^n
    \]
    Requiere resolver un sistema lineal en cada paso temporal.
\end{itemize}

\section{Evaluación con parámetros dados}

Se evaluarán las funciones implementadas anteriormente con los siguientes parámetros:
\[
\alpha = 1, \quad \Delta t = 0.001, \quad \Delta x = 0.05, \quad T = 0.1
\]

La condición inicial $f(x)$ utilizada es:
\[
f(x) =
\begin{cases}
4x & 0 \leq x < \tfrac{1}{4}, \\
-2x + \tfrac{3}{2} & \tfrac{1}{4} \leq x < \tfrac{1}{2}, \\
2x - \tfrac{1}{2} & \tfrac{1}{2} \leq x < \tfrac{3}{4}, \\
-4x + 4 & \tfrac{3}{4} \leq x \leq 1.
\end{cases}
\]

% \begin{figure}[H]
% \centering
% \includegraphics[width=0.7\textwidth]{condicion_inicial.png}
% \caption{Condición inicial $u(x,0) = f(x)$.}
% \end{figure}

\section{Visualización y análisis de resultados}

Con los valores obtenidos en las simulaciones se construirá un GIF para visualizar
cómo varía la solución a medida que evoluciona el tiempo. Se analizarán las diferencias entre los métodos explícito e implícito y se discutirá cómo el parámetro $r$ afecta la estabilidad y precisión de los resultados.

\newpage

\part{Ecuación de Transporte}

\section{Formulación matemática - Derivación de los métodos explícito e implícito}

En esta sección vamos a aplicar el mismo procedimiento que en la parte 1 - ecuación del calor, pero ahora sobre la ecuación de transporte con velocidad constante $\alpha$:
\[
\frac{\partial u}{\partial t} + \alpha \frac{\partial u}{\partial x} = 0,
\quad x \in (0,1), \quad t > 0
\]

con condición inicial sugerida:
\[
u(x,0) = f(x) = \sin(\pi x).
\]

Aquí $u(x,t)$ describe cómo una cantidad se desplaza en el tiempo a lo largo del eje $x$ con velocidad $\alpha$. \\

Al igual que antes, trabajamos con condiciones de frontera de Dirichlet homogéneas:
\[
u(0,t)=u(1,t)=0
\]

\subsection{Discretización: malla de espacio y tiempo}

La malla espacial y temporal es exactamente la misma que usamos en la ecuación del calor:

\begin{itemize}
    \item Puntos espaciales:
    \[
    x_j = j \Delta x, \quad j = 0,1,\dots,N, \quad \Delta x = \tfrac{1}{N}.
    \]

    \item Puntos temporales:
    \[
    t_n = n\Delta t, \quad n = 0,1,2,\dots,M.
    \]
\end{itemize}

Denotamos la aproximación como $u_j^n \approx u(x_j,t_n)$.

\subsection{Aproximación de las derivadas con diferencias finitas}

Como en la ecuación del calor, aproximamos las derivadas con fórmulas en diferencias finitas.  
La diferencia es que aquí sólo necesitamos la derivada espacial de primer orden.

\begin{itemize}
    \item Derivada temporal: diferencia hacia adelante
    \[
    \frac{\partial u}{\partial t}(x_j,t_n) \approx \frac{u_j^{n+1}-u_j^n}{\Delta t}.
    \]

    \item Derivada espacial: diferencia hacia atrás (esquema para $\alpha>0$)
    \[
    \frac{\partial u}{\partial x}(x_j,t_n) \approx \frac{u_j^n - u_{j-1}^n}{\Delta x}.
    \]
\end{itemize}

\subsection{Método explícito}

Partimos de la ecuación discretizada:
\[
\frac{u_j^{n+1}-u_j^n}{\Delta t} + \alpha \frac{u_j^n - u_{j-1}^n}{\Delta x} = 0.
\]

\subsection*{Definición de $r$:}
Sustituimos $r = \tfrac{\alpha \Delta t}{\Delta x}$:
\[
u_j^{n+1}-u_j^n = -r\,(u_j^n - u_{j-1}^n).
\]

\paragraph{Paso 1. Expandir}  
\[
u_j^{n+1} = u_j^n - r\,u_j^n + r\,u_{j-1}^n.
\]

\paragraph{Paso 2. Agrupar términos}  
\[
u_j^{n+1} = (1-r)\,u_j^n + r\,u_{j-1}^n.
\]

\paragraph{Esquema explícito}  
\[
\boxed{u_j^{\,n+1} = (1-r)\,u_j^n + r\,u_{j-1}^n}
\]

\subsection*{Condición de estabilidad:}
El esquema explícito es estable si y sólo si:
\[
0 \leq r \leq 1.
\]

\subsection*{Interpretación:}
El nuevo valor en $j$ es un promedio ponderado entre el valor anterior en $j$ y el vecino izquierdo $j-1$.  
Esto refleja el transporte de la información en la dirección de la velocidad $\alpha$.

\subsection{Método implícito}

Ahora evaluamos la derivada espacial en el tiempo $n+1$:
\[
\frac{u_j^{n+1}-u_j^n}{\Delta t} + \alpha \frac{u_j^{n+1} - u_{j-1}^{n+1}}{\Delta x} = 0.
\]

\paragraph{Paso 1. Sustituir $r$}  
\[
u_j^{n+1} - u_j^n = -r\,(u_j^{n+1} - u_{j-1}^{n+1}).
\]

\paragraph{Paso 2. Reorganizar}  
\[
(1-r)\,u_j^{n+1} + r\,u_{j-1}^{n+1} = u_j^n.
\]

\paragraph{Esquema implícito}  
\[
\boxed{(1-r)\,u_j^{\,n+1} + r\,u_{j-1}^{\,n+1} = u_j^n}
\]

\subsection*{Interpretación:}
En este caso, los valores de $u^{n+1}$ aparecen acoplados entre sí.  
Esto obliga a resolver un sistema lineal en cada paso temporal.  
La ventaja es que el método implícito es \textbf{incondicionalmente estable}, sin restricciones sobre $r$.

\subsection{Resumen comparativo}

\begin{itemize}
    \item \textbf{Explícito:}  
    \[
    u_j^{n+1} = (1-r)\,u_j^n + r\,u_{j-1}^n,
    \]
    sencillo de implementar, pero estable sólo si $0 \leq r \leq 1$.

    \item \textbf{Implícito:}  
    \[
    (1-r)\,u_j^{n+1} + r\,u_{j-1}^{n+1} = u_j^n,
    \]
    más costoso computacionalmente, porque requiere resolver un sistema lineal en cada paso, pero estable para cualquier $r$.
\end{itemize}

\section{Sistemas lineales y forma matricial}

\subsection{Recordemos el dominio}
Como ya vimos en la parte 1, trabajamos con $N+1$ puntos espaciales.  
Los valores desconocidos corresponden a los nodos internos $j=1,\dots,N-1$. En cada instante de tiempo usamos un vector de tamaño $N-1$.

\subsection{Método explícito como sistema lineal}

El método explícito se escribe como:
\[
u_j^{n+1} = (1-r)\,u_j^n + r\,u_{j-1}^n.
\]

\subsection*{Forma matricial:}
\[
U^{n+1} = B\,U^n,
\]
con
\[
U^n = 
\begin{bmatrix}
u_1^n \\[4pt] u_2^n \\[4pt] \vdots \\[4pt] u_{N-2}^n \\[4pt] u_{N-1}^n
\end{bmatrix}.
\] \\

La matriz $B$ en este caso es bidiagonal:
\[
B = 
\begin{bmatrix}
1-r & 0   & 0   & \cdots & 0 \\[4pt]
r   & 1-r & 0   & \cdots & 0 \\[4pt]
0   & r   & 1-r & \cdots & 0 \\[4pt]
\vdots & \vdots & \vdots & \ddots & 0 \\[4pt]
0   & 0   & 0   & r      & 1-r
\end{bmatrix}.
\]

\subsection*{Interpretación:}
La evolución temporal se obtiene simplemente multiplicando $U^n$ por $B$. Cada fila refleja la dependencia entre un nodo y su vecino izquierdo.

\subsection{Método implícito como sistema lineal}

El método implícito es:
\[
(1-r)\,u_j^{n+1} + r\,u_{j-1}^{n+1} = u_j^n.
\]

\subsection*{Forma matricial:}
\[
A\,U^{n+1} = U^n,
\]
donde
\[
A = 
\begin{bmatrix}
1-r & 0   & 0   & \cdots & 0 \\[4pt]
r   & 1-r & 0   & \cdots & 0 \\[4pt]
0   & r   & 1-r & \cdots & 0 \\[4pt]
\vdots & \vdots & \vdots & \ddots & 0 \\[4pt]
0   & 0   & 0   & r      & 1-r
\end{bmatrix}.
\]

\subsection*{Interpretación:}
Aquí $U^{n+1}$ está en ambos lados de la ecuación, por lo que en cada paso debemos resolver:
\[
U^{n+1} = A^{-1} U^n.
\]

\subsection{Resumen comparativo}

\begin{itemize}
    \item \textbf{Explícito:}  
    \[
    U^{n+1} = B\,U^n,
    \]
    evolución mediante una multiplicación matricial (bidiagonal).

    \item \textbf{Implícito:}  
    \[
    A\,U^{n+1} = U^n,
    \]
    requiere resolver un sistema lineal en cada paso temporal (también con matriz bidiagonal).
\end{itemize}

\section{Evaluación con parámetros dados}

Se evaluarán las funciones implementadas anteriormente con los siguientes parámetros:
\[
\alpha = 1, \quad \Delta t = 0.001, \quad \Delta x = 0.05, \quad T = 0.1
\]

La condición inicial $f(x)$ utilizada es:
\[
f(x) =
\begin{cases}
4x & 0 \leq x < \tfrac{1}{4}, \\
-2x + \tfrac{3}{2} & \tfrac{1}{4} \leq x < \tfrac{1}{2}, \\
2x - \tfrac{1}{2} & \tfrac{1}{2} \leq x < \tfrac{3}{4}, \\
-4x + 4 & \tfrac{3}{4} \leq x \leq 1.
\end{cases}
\]

% \begin{figure}[H]
% \centering
% \includegraphics[width=0.7\textwidth]{condicion_inicial.png}
% \caption{Condición inicial $u(x,0) = f(x)$.}
% \end{figure}

\section{Visualización y análisis de resultados}

Con los valores obtenidos en las simulaciones se construirá un GIF para visualizar
cómo varía la solución a medida que evoluciona el tiempo. Se analizarán las diferencias entre los métodos explícito e implícito y se discutirá cómo el parámetro $r$ afecta la estabilidad y precisión de los resultados.

\end{document}