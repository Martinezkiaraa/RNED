\documentclass[12pt,a4paper]{article}

% ======== PAQUETES ========
\usepackage[spanish]{babel}     % Idioma español
\usepackage[utf8]{inputenc}     % Acentos
\usepackage[T1]{fontenc}        % Caracteres correctos
\usepackage{amsmath, amssymb}   % Matemática avanzada
\usepackage{graphicx}           % Insertar imágenes
\usepackage{float}              % Control de posición de figuras
\usepackage{hyperref}           % Hipervínculos
\usepackage{geometry}           % Márgenes
\usepackage{listings}           % Código fuente
\usepackage{xcolor}             % Colores para código
\usepackage{caption}            % Mejoras en captions
\usepackage{subcaption}         % Subfiguras
\usepackage{booktabs}           % Tablas bonitas

% Configuración de márgenes
\geometry{top=2.5cm, bottom=2.5cm, left=2.5cm, right=2.5cm}

% Configuración del código fuente
\lstset{
    language=Python,
    backgroundcolor=\color{black!5},
    basicstyle=\ttfamily\footnotesize,
    keywordstyle=\color{blue},
    commentstyle=\color{green!50!black},
    stringstyle=\color{orange},
    numbers=left,
    numberstyle=\tiny,
    stepnumber=1,
    frame=single,
    breaklines=true,
    tabsize=4
}

% Datos de la portada
\title{\textbf{Métodos Computacionales - Trabajo Práctico 1}\\
Resoluciones Numéricas de Ecuaciones Diferenciales}
\author{
    Mariño Martina, Martinez Kiara\\ 
    Universidad Torcuato Di Tella
}
\date{\today}

% ======== DOCUMENTO ========
\begin{document}

% ======== PORTADA ========
\maketitle
\thispagestyle{empty}
\newpage

% ======== ÍNDICE ========
\tableofcontents
\newpage

% ======== SECCIONES ========

\section{Ecuación del Calor}
\subsection{Formulación matemática - Derivación de los métodos explícito e implícito}

En esta sección vamos a ver cómo se llega a las fórmulas de los métodos explícito e implícito para resolver la ecuación del calor usando diferencias finitas. La idea es empezar de la ecuación original, discretizarla, y después aproximar las derivadas.

\paragraph{La ecuación que queremos resolver,}

la ecuación del calor en una dimensión es:
\[
\frac{\partial u}{\partial t} = \alpha \frac{\partial^2 u}{\partial x^2},
\quad x \in (0,1), \quad t>0
\]

Donde:
\begin{itemize}
    \item $u(x,t)$ representa la temperatura en el punto $x$ y tiempo $t$.
    \item $\alpha > 0$ es la constante de difusión térmica.
\end{itemize}

Además, tenemos condiciones de frontera de Dirichlet:
\[
u(0,t) = 0, \quad u(1,t) = 0
\]
y una condición inicial que nos da la distribución de temperatura al inicio:
\[
u(x,0) = f(x).
\]

En palabras, se trata de una barra de longitud 1 donde los extremos se mantienen a temperatura cero. Sabemos cómo estaba la temperatura al comienzo y queremos ver cómo cambia con el tiempo.

\paragraph{Discretización: malla de espacio y tiempo.}

Para resolver el problema de forma numérica dividimos el espacio y el tiempo en puntos separados por pasos fijos:
\begin{itemize}
    \item Dividimos el intervalo espacial $[0,1]$ en $N$ puntos con separación $\Delta x$. Así, las posiciones son:
    \[
    x_j = j\Delta x, \quad j = 0,1,\dots,N
    \]
    donde $\Delta x = \frac{1}{N}$.
    \item El tiempo se divide en pasos de tamaño $\Delta t$. Los instantes de tiempo quedan como:
    \[
    t_n = n\Delta t, \quad n = 0,1,2,\dots,M
    \]
    con $M\Delta t = T$ siendo el tiempo final de simulación.
\end{itemize}

Notación: llamamos $u_j^n$ a la aproximación numérica de $u(x_j,t_n)$.  
Los nodos de frontera son $j=0$ y $j=N$ (donde ya conocemos $u$ gracias a las condiciones de borde). Los nodos internos son $j=1,\dots,N-1$, que son los que vamos a actualizar.

\paragraph{Aproximación de las derivadas con diferencias finitas.}

Vamos a reemplazar las derivadas por aproximaciones usando diferencias finitas:
\begin{itemize}
    \item Para la derivada temporal usamos una \textbf{diferencia hacia adelante}:
    \[
    \frac{\partial u}{\partial t}(x_j,t_n) \approx \frac{u_j^{n+1}-u_j^n}{\Delta t}.
    \]
    \item Para la segunda derivada espacial usamos \textbf{diferencias centradas}:
    \[
    \frac{\partial^2 u}{\partial x^2}(x_j,t_n) \approx \frac{u_{j+1}^n - 2u_j^n + u_{j-1}^n}{(\Delta x)^2}.
    \]
\end{itemize}

Esto introduce un error, pero mientras $\Delta x$ y $\Delta t$ sean pequeños, la aproximación es bastante buena.

\paragraph{Sustitución en la ecuación original - Derivación explícito.}

Si reemplazamos las aproximaciones en la ecuación del calor, nos queda:
\[
\frac{u_j^{n+1}-u_j^n}{\Delta t} = \alpha \frac{u_{j+1}^n - 2u_j^n + u_{j-1}^n}{(\Delta x)^2}.
\]

Definimos un parámetro muy útil llamado $r$:
\[
r = \frac{\alpha \Delta t}{(\Delta x)^2}.
\]
Este parámetro combina la constante física \(\alpha\) con el paso de tiempo \(\Delta t\) y el paso espacial \(\Delta x\). 

Reemplazando esta definición en la ecuación anterior:
\[
u_j^{\,n+1}-u_j^{\,n} = r\big(u_{j+1}^{\,n}-2u_j^{\,n}+u_{j-1}^{\,n}\big).
\]

El siguiente objetivo es aislar \(u_j^{\,n+1}\) para que quede explícito. Sumamos \(u_j^{\,n}\) en ambos lados:
\[
u_j^{\,n+1} = u_j^{\,n} + r\big(u_{j+1}^{\,n}-2u_j^{\,n}+u_{j-1}^{\,n}\big).
\]

Expandimos el paréntesis:
\[
u_j^{\,n+1} = u_j^{\,n} + r\,u_{j+1}^{\,n} - 2r\,u_j^{\,n} + r\,u_{j-1}^{\,n}.
\]

Agrupando términos similares, en especial los que dependen de \(u_j^n\):
\[
u_j^{\,n+1} = (1 - 2r)u_j^{\,n} + r\,u_{j-1}^{\,n} + r\,u_{j+1}^{\,n}.
\]
\newline
Esta última ecuación muestra que el nuevo valor de temperatura en la posición \(j\) y tiempo \(n+1\) se calcula como una combinación lineal de:
\begin{itemize}
    \item El valor previo en la misma posición, \(u_j^n\), ponderado por \((1-2r)\).
    \item El vecino de la izquierda, \(u_{j-1}^n\), ponderado por \(r\).
    \item El vecino de la derecha, \(u_{j+1}^n\), también ponderado por \(r\).
\end{itemize}


Así, nos queda el esquema \textbf{explícito} completo:
\[
\boxed{u_j^{\,n+1} = (1 - 2r)u_j^{\,n} + r\,u_{j-1}^{\,n} + r\,u_{j+1}^{\,n}},
\]


\paragraph{Condición de estabilidad:}  
Este método solo funciona bien si se cumple:
\[
r = \frac{\alpha \Delta t}{(\Delta x)^2} \leq \frac{1}{2}.
\]
Si no se cumple, la solución explota y empieza a oscilar de manera irreal.

\paragraph{Interpretación:}  
La temperatura en $j$ tiende a ``suavizarse'' dependiendo de cómo están los vecinos $j-1$ y $j+1$. Es como un promedio que se va corrigiendo paso a paso.

\paragraph{Derivación Método implícito.}

\paragraph{Partimos de las aproximaciones}  
Usamos la misma aproximación temporal (diferencia hacia adelante en el numerador) pero evaluamos la segunda derivada espacial en el tiempo \(n+1\):
\[
\frac{u_j^{\,n+1}-u_j^{\,n}}{\Delta t}
\approx
\alpha\,\frac{u_{j+1}^{\,n+1}-2u_j^{\,n+1}+u_{j-1}^{\,n+1}}{(\Delta x)^2}.
\]

\paragraph{Paso 1. Multiplicar por \(\Delta t\)}  
Multiplicamos ambos lados por \(\Delta t\) para quitar el denominador:
\[
u_j^{\,n+1}-u_j^{\,n}
= \alpha\,\frac{\Delta t}{(\Delta x)^2}\big(u_{j+1}^{\,n+1}-2u_j^{\,n+1}+u_{j-1}^{\,n+1}\big).
\]

\paragraph{Paso 2. Definir \(r\) y sustituir}  
Como antes definimos
\[
r := \frac{\alpha\,\Delta t}{(\Delta x)^2},
\]
y reemplazamos:
\[
u_j^{\,n+1}-u_j^{\,n} = r\big(u_{j+1}^{\,n+1}-2u_j^{\,n+1}+u_{j-1}^{\,n+1}\big).
\]

\paragraph{Paso 3. Llevar todos los términos con \(u^{\,n+1}\) al mismo lado}  
Queremos agrupar las incógnitas de tiempo \(n+1\) a la izquierda y dejar el término conocido \(u_j^{\,n}\) a la derecha. Para eso pasamos el término de la derecha al lado izquierdo:
\[
u_j^{\,n+1} - r\big(u_{j+1}^{\,n+1}-2u_j^{\,n+1}+u_{j-1}^{\,n+1}\big) = u_j^{\,n}.
\]

\paragraph{Paso 4. Expandir el paréntesis}  
\[
u_j^{\,n+1} - r\,u_{j+1}^{\,n+1} + 2r\,u_j^{\,n+1} - r\,u_{j-1}^{\,n+1} = u_j^{\,n}.
\]

\paragraph{Paso 5. Agrupar coeficientes según cada incógnita}  
Agrupamos los coeficientes de \(u_{j-1}^{\,n+1}, u_j^{\,n+1}, u_{j+1}^{\,n+1}\):
\[
(-r)\,u_{j-1}^{\,n+1} + (1+2r)\,u_j^{\,n+1} + (-r)\,u_{j+1}^{\,n+1} = u_j^{\,n}.
\]

De forma más explícita:
\[
-r\,u_{j-1}^{\,n+1} + (1+2r)\,u_j^{\,n+1} - r\,u_{j+1}^{\,n+1} = u_j^{\,n}.
\]

\paragraph{Paso 6. Interpretación local}  
Para cada nodo interior \(j\) la ecuación relaciona al nuevo valor \(u_j^{\,n+1}\) con sus vecinos en el mismo nuevo tiempo \(n+1\). No hay una fórmula directa para \(u_j^{\,n+1}\) en términos solo de valores en \(n\); por eso hay que resolver un sistema.

\subsection{Sistemas lineales y forma matricial}
\paragraph{Recordemos el dominio:}
Tenemos \(N+1\) puntos en el espacio, numerados de \(j=0\) a \(j=N\).  
Los extremos (\(j=0\) y \(j=N\)) son conocidos por las condiciones de frontera \(u(0,t)=u(1,t)=0\).  
Los valores desconocidos están en los puntos internos \(j=1,2,\dots,N-1\).  
Eso significa que en cada instante de tiempo tenemos que trabajar con un vector de tamaño \(N-1\).

---

\subsubsection*{1. Método explícito como sistema lineal}

El método explícito tiene esta fórmula final:
\[
u_j^{n+1} = (1-2r)u_j^n + r\,u_{j-1}^n + r\,u_{j+1}^n, 
\quad j = 1, 2, \dots, N-1.
\]

Esto se puede escribir como un producto matricial:
\[
U^{n+1} = B\,U^n,
\]
donde:
\[
U^n = 
\begin{bmatrix}
u_1^n \\ u_2^n \\ \vdots \\ u_{N-2}^n \\ u_{N-1}^n
\end{bmatrix}
\]
es el vector con los valores de los nodos internos en el tiempo \(t_n\).

La matriz \(B\) tiene forma tridiagonal:
\[
B = 
\begin{bmatrix}
1-2r & r     & 0     & \cdots & 0 \\[4pt]
r     & 1-2r & r     & \cdots & 0 \\[4pt]
0     & r     & 1-2r & \cdots & 0 \\[4pt]
\vdots& \vdots& \vdots& \ddots & r \\[4pt]
0     & 0     & 0     & r      & 1-2r
\end{bmatrix}.
\]

\paragraph{Interpretación:}  
Esto significa que para pasar del tiempo \(t_n\) al tiempo \(t_{n+1}\), basta con multiplicar el vector actual \(U^n\) por la matriz \(B\).  
---

\subsubsection*{2. Método implícito como sistema lineal}

En el método implícito, la ecuación final para cada nodo es:
\[
-r\,u_{j-1}^{n+1} + (1+2r)\,u_j^{n+1} - r\,u_{j+1}^{n+1} = u_j^n,
\quad j = 1, 2, \dots, N-1.
\]

Esto se puede ver como un sistema lineal:
\[
A\,U^{n+1} = U^n,
\]
donde \(U^{n+1}\) es el vector de incógnitas en el nuevo tiempo, igual que antes:
\[
U^{n+1} = 
\begin{bmatrix}
u_1^{n+1} \\ u_2^{n+1} \\ \vdots \\ u_{N-2}^{n+1} \\ u_{N-1}^{n+1}
\end{bmatrix}.
\]

La matriz \(A\) también es tridiagonal, pero con valores diferentes:
\[
A = 
\begin{bmatrix}
1+2r & -r    & 0     & \cdots & 0 \\[4pt]
-r    & 1+2r & -r    & \cdots & 0 \\[4pt]
0     & -r    & 1+2r & \cdots & 0 \\[4pt]
\vdots& \vdots& \vdots& \ddots & -r \\[4pt]
0     & 0     & 0     & -r     & 1+2r
\end{bmatrix}.
\]

\paragraph{Interpretación:}  
En este caso, no podemos calcular \(U^{n+1}\) directamente porque está "mezclado" dentro de la ecuación.  
Para avanzar un paso en el tiempo hay que resolver el sistema lineal \(A\,U^{n+1}=U^n\).  
Esto se puede hacer en Python con:
\[
\texttt{U\_next = numpy.linalg.solve(A, U)}
\]

\section{Ecuación de transporte}

La ecuación es:
\[
\frac{\partial u}{\partial t} = a\,\frac{\partial u}{\partial x}, 
\quad x \in (0,1), \quad t > 0.
\]

\paragraph{1. Discretización del dominio}

Dividimos el espacio y el tiempo en pasos uniformes:
\begin{itemize}
    \item Posiciones: \(x_j = j\Delta x\), con \(j = 0,1,\dots,N\), donde \(\Delta x = \frac{1}{N}\).
    \item Tiempos: \(t_n = n\Delta t\), con \(n = 0,1,\dots,M\), donde \(M\Delta t = T\).
\end{itemize}

Como antes, usaremos la notación \(u_j^n \approx u(x_j, t_n)\).

\paragraph{Derivada temporal (forward):}

\[
\frac{\partial u}{\partial t}(x_j, t_n) \approx \frac{u_j^{n+1}-u_j^n}{\Delta t}.
\]

\paragraph{Derivada espacial:}  
Hay varias formas de aproximarla. Para la versión más simple usamos una \textbf{diferencia hacia atrás}:
\[
\frac{\partial u}{\partial x}(x_j, t_n) \approx \frac{u_j^n - u_{j-1}^n}{\Delta x}.
\]

Esto significa que el valor en \(j\) depende de él mismo y de su vecino de la izquierda.

Sustituimos las aproximaciones en la ecuación de transporte:
\[
\frac{u_j^{n+1}-u_j^n}{\Delta t} = a\,\frac{u_j^n - u_{j-1}^n}{\Delta x}.
\]

\paragraph{Definimos r nuevamente:}
\[
r := \frac{a\,\Delta t}{\Delta x}.
\]
Reemplazando \(r\) en la ecuación, queda:
\[
u_j^{n+1}-u_j^n = r\,(u_j^n - u_{j-1}^n).
\]

\paragraph{Método explícito}

Aislamos \(u_j^{n+1}\) para obtener la fórmula explícita:
\[
u_j^{n+1} = u_j^n + r\,(u_j^n - u_{j-1}^n).
\]

Si expandimos el paréntesis:
\[
u_j^{n+1} = (1+r)u_j^n - r\,u_{j-1}^n.
\]

\paragraph{Interpretación:}
El valor en la posición \(j\) y tiempo \(n+1\) depende del valor actual \(u_j^n\) y del vecino a la izquierda \(u_{j-1}^n\).  
Es como si la información se fuera "moviendo" hacia la derecha con velocidad \(a\).

\paragraph{Condición de estabilidad:}  
Para que la solución sea estable, el r debe cumplir:
\[
r = \frac{a\,\Delta t}{\Delta x} \leq 1.
\]
Si no se cumple, la simulación genera oscilaciones y se vuelve inestable.

\paragraph{Método implícito}

Ahora evaluamos la derivada espacial en el \textbf{tiempo futuro} \(n+1\).  
Partimos de:
\[
\frac{u_j^{n+1}-u_j^n}{\Delta t} = a\,\frac{u_j^{n+1} - u_{j-1}^{n+1}}{\Delta x}.
\]

Multiplicamos por \(\Delta t\) y reemplazamos \(r = \frac{a\,\Delta t}{\Delta x}\):
\[
u_j^{n+1}-u_j^n = r\,(u_j^{n+1}-u_{j-1}^{n+1}).
\]

Reordenamos para dejar todos los términos con \(u^{n+1}\) a la izquierda:
\[
u_j^{n+1} - r\,u_j^{n+1} + r\,u_{j-1}^{n+1} = u_j^n.
\]

Simplificamos términos semejantes:
\[
(1-r)u_j^{n+1} + r\,u_{j-1}^{n+1} = u_j^n.
\]
\subsection{Sistemas Lineales y Forma Matricial}
\subsubsection*{1. Método implícito como sistema lineal}
Agrupamos todas las incógnitas \(u_1^{n+1}, \dots, u_{N-1}^{n+1}\) en un vector:
\[
U^{n+1} = 
\begin{bmatrix}
u_1^{n+1}\\
u_2^{n+1}\\
\vdots\\
u_{N-1}^{n+1}
\end{bmatrix}.
\]

Así, la ecuación implícita se puede escribir como un sistema lineal:
\[
A\,U^{n+1} = U^n,
\]
donde \(A\) es una matriz \textbf{triangular inferior}:
\[
A =
\begin{bmatrix}
1-r & 0   & 0   & \cdots & 0 \\[4pt]
c   & 1-r & 0   & \cdots & 0 \\[4pt]
0   & c   & 1-r & \cdots & 0 \\[4pt]
\vdots & \vdots & \vdots & \ddots & 0 \\[4pt]
0   & 0   & 0   & c      & 1-r
\end{bmatrix}.
\]

En cada paso temporal hay que resolver este sistema para encontrar \(U^{n+1}\).  
\subsubsection*{2. Método explícito como sistema lineal}

El método explícito tiene esta fórmula final:

\[
u_j^{n+1} = (1+r)u_j^n - r\,u_{j-1}^n.
\]

Esto se puede escribir como un producto matricial:
\[
U^{n+1} = B\,U^n,
\]
donde:
\[
U^n = 
\begin{bmatrix}
u_1^n \\ u_2^n \\ \vdots \\ u_{N-2}^n \\ u_{N-1}^n
\end{bmatrix}
\]
es el vector con los valores de los nodos internos en el tiempo \(t_n\).

La matriz \(B\) tiene forma tridiagonal:
\[
B = 
\begin{bmatrix}
1-2r & r     & 0     & \cdots & 0 \\[4pt]
r     & 1-2r & r     & \cdots & 0 \\[4pt]
0     & r     & 1-2r & \cdots & 0 \\[4pt]
\vdots& \vdots& \vdots& \ddots & r \\[4pt]
0     & 0     & 0     & r      & 1-2r
\end{bmatrix}.
\]

\paragraph{Interpretación:}  
Esto significa que para pasar del tiempo \(t_n\) al tiempo \(t_{n+1}\), basta con multiplicar el vector actual \(U^n\) por la matriz \(B\).  
---


\subsection{Discusión}
Comparando los métodos explícito e implícito se observa que\ldots

\section{Ecuación de Transporte}
\subsection{Formulación}
La ecuación de transporte está dada por:
\begin{equation}
    \frac{\partial u}{\partial t} = a \frac{\partial u}{\partial x}, \quad
    u(x,0) = \sin(\pi x)
\end{equation}

\subsection{Resultados y análisis}
% Puedes insertar gráficos y explicaciones aquí

\section{Conclusiones}
En este trabajo se implementaron métodos numéricos para resolver ecuaciones diferenciales parciales. 
Se concluye que\ldots

\end{document}
